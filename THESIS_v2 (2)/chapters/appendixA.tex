\chapter{Stages Management}
\label{chap:stage}

% \section{Stages Management}
To ensure that the problem-solving process is carried out in the correct order of steps, each classmate agent will receive input as the name, specific description of each stage and the goal to be achieved to ensure that the whole group completes all tasks before moving on. This ensures that the process is structured and that students effectively achieve knowledge after discussion. Table \ref{tab:stages} outlines the four stages in detail, including the example prompts provided to the agent at each stage to guide the discussion.

This framework is adaptable to various educational contexts, such as a project-based science experiment with distinct stages—identifying a problem, researching, designing a solution, building and testing—each tailored to foster critical thinking; or a student debate on a societal issue with stages including argument construction, evidence evaluation, and synthesis of conclusions. Its flexible design accommodates user preferences while ensuring a systematic and comprehensive process to optimize learning outcomes.


{\footnotesize

\begin{table}[H] % hoặc [htbp] tuỳ bạn
\centering
\caption{Four stages in the mathematical problem}
\label{tab:stages}

\begin{tabular}{| p{1cm} | p{2.5cm} | p{7cm} | p{5cm} |}
    \hline
    \textbf{Stage} & \textbf{Name} & \textbf{Description} & \textbf{Goal}\\
    \hline
    1
    &
    Understanding the Problem
    &
    This initial phase involves a thorough analysis aimed at clarifying the conditions and the conclusion of the problem. It requires examining all statements provided, distinguishing between known and unknown elements, and identifying what must be calculated, proven, or constructed. 
    
    \textit{(Ví dụ trong prompt: Tìm hiểu bài toán cho những gì? Đâu là ẩn? Đâu là dữ liệu? và Bài toán yêu toán yêu cầu tìm hay chứng minh điều gì?)}
    &
    At this stage, students are required to grasp the information provided and identify the problem. 
    
    \textit{(Ví dụ trong prompt: Nhận biết đây là dạng bài toán xét tính đơn điệu của hàm số bậc nhất trên bậc nhất.)}
    \\
    \hline

    2
    &
    Devising a Plan
    &
    From the information of the problem, students will synthesize the necessary data to solve the problem, propose a method to solve the problem, consider whether the solution method satisfies the data or conditions given in the problem. 
    
    \textit{(Ví dụ trong prompt:Đề xuất phương pháp giải bài từ quan sát đánh giá bài toán. Nhận xét, phân tích một phương pháp cụ thể xem khả thi không.)}
    &
    After evaluating and analyzing, students choose the most popular and suitable method. 
    
    \textit{(Ví dụ trong prompt: Thống nhất được cách làm phổ biến nhất là dùng đạo hàm để xét tính đơn điệu và vẽ bảng biến thiên.)}
    \\
    \hline
    3
    &
    Carry out the Plan
    &
    Students will rely on the method found combined with mathematical language and symbols to present a solution. 
    \textit{(Ví dụ trong prompt: Thực hiện cụ thể, lần lượt từng bước làm theo kế hoạch: Bước 1: \ldots)}
    &
    Students understand how to do the exercise according to the chosen method, ensuring all steps are completed. 
    \textit{(Ví dụ trong prompt: Tính đạo hàm và xét dấu đúng; Nhận biết tính đơn điệu; Vẽ bảng biến thiên)}
    \\
    \hline

    4
    &
    Looking back
    &
    This phase encourages reflective inquiry after solving the problem to reinforce understanding and extend learning.
    
    \textit{(Ví dụ trong prompt: Tóm tắt những bước chính đã làm, Rút ra được nguyên tắc làm bài.)}
    &
    Students can present their experiences in doing exercises and apply them to similar types. 
    
    \textit{(Ví dụ trong prompt: Rút ra được nguyên tắc làm dạng này như sau: Bước 1: Tìm tập xác định; \ldots ;Bước 4: Nêu kết luận\ldots)}
    \\
    \hline
\end{tabular}
\end{table}
}



