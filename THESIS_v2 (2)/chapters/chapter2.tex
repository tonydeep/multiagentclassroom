% \begin{savequote}[75mm] 
% This is some random quote to start off the chapter.
% \qauthor{Firstname lastname} 
% \end{savequote}

\chapter{LITERATURE REVIEW}

% \newthought{Lorem ipsum dolor sit amet}, consectetuer adipiscing elit. Morbi commodo, ipsum sed pharetra gravida, orci magna rhoncus neque, id pulvinar odio lorem non turpis. Nullam sit amet enim. Suspendisse id velit vitae ligula volutpat condimentum. Aliquam erat volutpat. Sed quis velit. Nulla facilisi. Nulla libero. Vivamus pharetra posuere sapien. Nam consectetuer. Sed aliquam, nunc eget euismod ullamcorper, lectus nunc ullamcorper orci, fermentum bibendum enim nibh eget ipsum. Donec porttitor ligula eu dolor. Maecenas vitae nulla consequat libero cursus venenatis. Nam magna enim, accumsan eu, blandit sed, blandit a, eros.

\section{Generative AI in Education}
\subsection{Before the Era of Large Language Models}
Prior to the widespread adoption of large language models (LLMs) around 2018–2019, artificial intelligence had already made substantial contributions to education through technologies such as Intelligent Tutoring Systems (ITS), Machine Learning (ML), Natural Language Processing (NLP), and intent-based classification systems. ITS \cite{WikipediaITS} emerged in the 1970s and evolved into advanced, knowledge-centric platforms by the 1990s, which provided tailored instruction by adapting to the unique needs of each student. Similarly, ML techniques have been applied to educational data: they identify at-risk students, personalize instructional content, and even automate routine tasks such as grading through automated essay-scoring tools. These algorithmic advances not only flag struggling learners early but also enable automated personalized interventions, thereby establishing a robust foundation for adaptive, efficient educational practice. Concurrently, NLP-driven applications facilitated more natural language interaction in learning: automated scoring and writing-assistance tools provide scalable feedback on student writing, while conversational agents (chatbots) use intent classification to interpret student questions and deliver relevant responses, improving engagement and real-time support.

\subsection{During the Era of Large Language Models}
The advent of Large Language Models (LLMs) has marked a significant turning point in the educational domain, with their influence becoming particularly notable since the release of ChatGPT in 2022. This change in mainstream architecture has moved the design-space from old NLP methods to transformer-based, state-of-the-art models that can do things we could never do before. LLMs are now more accessible giving educators and students the opportunity to take advantage of LLMs for various educational purposes. One of the cornerstone benefits of LLMs is their scalability \cite{wang2024large}, which allows them to deliver highly personalized learning experiences to diverse student populations. By automating labor-intensive tasks such as generating educational content, providing real-time feedback, and even grading assignments, LLMs make it feasible to tailor instruction to individual needs on a large scale. 

In addition to their analytical and creative applications, LLMs bring a unique interactive dimension to education through their roleplay capabilities. By simulating characters—ranging from historical figures \cite{zhu2025exploring} to conversational partners—LLMs create immersive learning environments that enhance engagement and practical skill. This interactivity not only makes learning more enjoyable but also bridges theoretical knowledge with real-world application. Additionally, certain studies employ LLMs to mimic real students, such as evaluating whether the error rate in LLM-solved multiple-choice questions aligns with that of students \cite{2025arXiv250215140L}, thus enabling the creation of high-quality multiple-choice questions.


\subsection{Some Related Products for Education}
Globally, a diverse range of products enhances educational outcomes:
\begin{itemize}
  \item Speechify \cite{speechify2025}: This tool utilizes generative AI to convert text into natural speech, significantly improving accessibility for students with learning disabilities such as dyslexia or visual impairments.
  \item NOLEJ \cite{nolej}: Designed for educators and instructional designers, NOLEJ uses generative AI to streamline the creation of e-learning content. It can rapidly generate interactive lessons, quizzes, and multimedia resources from raw educational material.
  \item Grammarly \cite{grammarly}: Beyond basic grammar and spell-checking, Grammarly’s generative AI features can suggest style improvements, tone adjustments, and even full sentence rephrasings, etc.
\end{itemize}
These tools illustrate the broad applications of generative AI, from content generation to creative visualization, transforming how education is delivered worldwide.

The market for generative AI in education is projected to grow substantially, reaching USD 7,701.9 million by 2033 \cite{market2024education} with a compound annual growth rate (CAGR) of 39.5\%, fueled by demand for personalized learning solutions. Usage statistics reveal widespread adoption: 44\% of children engage with generative AI for schoolwork, and 60\% of teachers integrate it into their teaching practices. However, ethical challenges persist, with 24.11\% of charter high school students reporting AI-related cheating incidents, highlighting the need to address misuse alongside its benefits. 

\section{One-to-one Tutoring}
\subsection{Pedagogical Strategies in One-to-One Tutoring}

One-to-one tutoring using AI systems, especially those powered by LLMs, leverages various pedagogical strategies to enhance learning outcomes. These strategies are designed to mimic human tutoring while scaling to meet individual student needs, including:
\begin{itemize}
  \item Socratic Method: This approach involves AI tutors asking probing, open-ended questions to stimulate critical thinking and guide students toward self-discovery. The leading artificial intelligence company Anthropic \cite{anthropic2025} has launched an educational product that underscores the importance of this methodology.
  \item Inquiry-Based Learning \cite{gousopoulos2024developing}: This strategy encourages students to pose questions and explore topics through guided investigation, promoting active learning. This method is learner-centered, fostering curiosity and deeper engagement, particularly in general education settings.
  \item Personalized Learning \cite{razafinirina2024pedagogical}: Tailoring educational content to individual student needs, preferences, and learning styles is a cornerstone of AI-driven tutoring. This strategy enhances engagement and motivation, with chatbots analyzing student data to adapt teaching methods, etc.
\end{itemize}

\subsection{Applications}
Educational chatbots are applied across a wide range of subjects and educational fields \cite{chu2025llm}, demonstrating their versatility:
\begin{itemize}
  \item STEM Subjects
  \item English as a Foreign Language (EFL)
  \item Programming
  \item Medical Education
  \item Soft Skills Development
\end{itemize}

\subsection{Beyond One-to-One}
While one-to-one tutoring offers personalized attention, it faces challenges in simulating the full spectrum of classroom interactions, particularly social and collaborative elements. One-to-one settings often miss peer learning opportunities, which are crucial for social development and collaborative skills. In contrast, traditional classrooms foster peer interactions that enhance learning through discussion and shared problem-solving. These limitations highlight the need for a more comprehensive approach to simulate realistic learning experiences.

\section{Virtual Classroom}
\subsection{Motivation}
The motivation behind designing a Virtual Classroom comes from the need to create a more natural and interactive educational environment that mirrors the complexities of human conversation. By integrating multiple AI agents, the virtual classroom can create a collaborative environment in which these agents actively engage alongside students. This design aims to shift away from tutor-centric models, encouraging students’ proactive involvement and enhancing the overall learning experience through sophisticated interaction management.

For students and tutors, this system offers significant benefits by promoting active participation and providing practical training opportunities. Students gain from AI agents acting as peers or tutors, which can enhance engagement through emotional support and equality in collaborative learning activities, as seen in systems like SimClass \cite{SimClass}. These agents enrich peer conversations and support knowledge co-construction, making learning more interactive and supportive. For novice tutors, a virtual classroom with AI-driven scenarios, inspired by TutorUp \cite{pan2025tutorup}, provides a cost-effective alternative to traditional training methods. It enables them to practice teaching through classroom simulations that might be encountered in online teaching,  such as dealing with inattentive students.

Integrating virtual reality (VR) \cite{liu2024classmeta} with multiple AI agents will further highlight the potential of the virtual classroom by creating an immersive, socially interactive learning space. Research indicates that VR classrooms foster social interaction and boost learner motivation through collaborative social presence, where participants’ actions influence one another in real-time. By combining VR’s immersive capabilities with AI agents that can engage naturally with students and each other, the system simulates authentic educational settings more effectively. This not only increases student interest but also offers a scalable, accessible solution for both learning and tutor training.

\subsection{Multi-agent Systems (MAS)}
LLM-based multi-agent systems \cite{wang2024survey} involve multiple intelligent agents powered by large language models (LLMs) collaborating to address complex tasks. These systems enhance capabilities such as planning, decision-making, memory, and tool use compared to single-agent setups. For example, in software development, one agent might focus on design while another handles coding, improving efficiency through task distribution. Agents employ structured communication methods—cooperative, debate, or competitive—to refine solutions, which facilitate coordination and task execution. Memory mechanisms, including short-term for recent interactions and long-term for historical data, allow agents to maintain context and improve over time. Additionally, tool integration, such as APIs or code interpreters, extends their functionality to domains like robotics and economic simulations.

In Virtual Classrooms, MAS are highly suitable because they can simulate a dynamic, interactive educational environment akin to a real classroom. Multiple AI agents can assume roles such as instructors, assistants, or peer learners, interacting with real students to provide personalized feedback and foster collaborative learning through simulated discussions. This setup leverages the agents’ ability to offer diverse perspectives and adapt to individual student needs, enhancing engagement and understanding. By mimicking group dynamics and facilitating interactive problem-solving, MAS create a rich, adaptive learning experience that mirrors traditional classroom interactions, making them an effective tool for education.

\subsection{Turn-takings in Multi-Party Conversations}
Studies such as SimClass \cite{SimClass}  and MathVC \cite{MathVC} have proposed approaches to managing the turns of conversation between agents. This method is based on conversation data and descriptions of the roles of agents to select the most suitable person to talk. Although it is also effective, it can be seen that this approach leaves agents in a passive position when they are selected by another manager agent. In reality, when people talk to each other, they will think independently before speaking. Therefore, a more comprehensive solution is needed to simulate this multi-participant conversation to increase the naturalness of communication and encourage users to participate more actively and comfortably in the conversation, thereby expecting to increase the learning outcomes of users.

\section{Summary and Research Gap}
AI in education has undergone significant evolution, progressing from early technologies like Intelligent Tutoring Systems (ITS) and Machine Learning (ML) to the transformative era of Large Language Models (LLMs). These advancements have enabled scalable personalized learning, enhanced creativity, and interactive roleplay scenarios, revolutionizing how educational content is delivered and experienced. However, while one-to-one tutoring powered by LLMs offers tailored pedagogical strategies, it falls short in replicating the collaborative dynamics inherent in a traditional classroom setting. To address this limitation, the concept of a Virtual Classroom using multi-agent systems (MAS) has emerged as a promising solution.

Despite these advancements, there are notable challenges in the study of Virtual Classrooms simulated by multi-agent systems. The concept is relatively new, with few studies exploring similar ideas, such as SimClass and MathVC, indicating that the field is still underexplored. Additionally, evaluating the effectiveness of such systems presents significant challenges. Unlike traditional one-to-one tutoring, assessing a Virtual Classroom requires analyzing complex interactions among multiple AI agents and students. This complexity makes it difficult to measure learning outcomes, engagement, and collaboration. Nevertheless, these challenges present opportunities for further development.

