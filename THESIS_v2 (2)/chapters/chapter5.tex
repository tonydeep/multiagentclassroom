
\chapter{CONCLUSION}

\section{Limitations}
Similar to other LLM agents, this system encounters challenges with token costs and latency, as it demands multiple LLM queries to produce each agent's response. Test data is limited in both quantity and quality, leading to results that may not fully reflect the system’s performance. In addition, the system's behavior is still heavily dependent on prompt design, making it sensitive to small changes in wording or formatting. This can affect consistency and robustness across different use cases. Furthermore, the lack of long-term memory across sessions limits the agents' ability to model deeper, ongoing collaboration or learning over time.

\section{Ethical Considerations}
Although the integration of AI agents in educational environments can enrich the learning experience, they cannot fully replace the essential functions of human teachers in cultivating students’ skills. Similarly, the presence of real peers is critical for the development of social interaction, group identity, and self-esteem. Consequently, the implementation of such systems requires more comprehensive and interdisciplinary research, particularly informed by insights from psychology, pedagogy, and related disciplines.

\section{Future Works}
Future developments could introduce more diverse forms of classroom interactions by drawing from various educational settings, while also incorporating emerging technologies to enrich the learning experience. For example, techniques like question generation and knowledge tracing could be utilized to further personalize the agents’ responses based on individual student needs.

It is crucial to conduct further experiments to explore the impact of these systems on real student experiences, assessing how they interact with multi-agent environments and identifying ways to improve the overall user experience in diverse educational settings. Future research should incorporate behavioral studies to observe how different groups adjust an agent’s behavior. Understanding which aspects of an agent’s behavior are most frequently modified by different group types could allow for the creation of a prioritized ranking of behavioral controls, providing valuable insights into how to tailor AI agents to different collaborative and educational settings.

