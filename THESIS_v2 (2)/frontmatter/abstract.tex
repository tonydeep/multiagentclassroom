\abstractpage{

\begin{center}
    \Large \textbf{ABSTRACT} \normalsize\\[13pt]
\end{center}

% Nội dung abstract bằng tiếng Anh
{\fontsize{12}{14}\selectfont
Traditional educational environments, especially in mathematics, often struggle to provide adequate personalized support for every student. With the rapid advancement of Generative AI, students now have the opportunity to access knowledge more deeply through interaction with AI-powered chatbots. However, most existing systems still position AI primarily as an individual tutor, without fully leveraging the social dynamics inherent in traditional classroom settings. This highlights a promising research direction: simulating multi-participant virtual classrooms—where students not only acquire knowledge but also develop essential communication and collaboration skills. Multi-Agent Systems (MAS) based on Large Language Models (LLMs) have emerged as a potential solution to this challenge, thanks to their capabilities in reasoning, decision-making, and flexible coordination among agents. In a virtual classroom context, agents can be designed with various roles such as classmates or teachers, collaborating with real students toward shared learning goals. Nevertheless, simulating a multi-agent classroom presents significant challenges, particularly in managing turn-taking and ensuring natural, coherent dialogue. This thesis presents a framework for designing pedagogically grounded learning agents, with a specific focus on mathematical modeling. Furthermore, it introduces a novel mechanism for managing conversational turn-taking, which improves coherence and interaction quality compared to previous methods. Experimental results and evaluations demonstrate that the agents not only fulfill their assigned roles but also proactively determine appropriate moments to speak, contributing to an interactive, adaptive, and engaging learning experience for students.

\textit{\textbf{Keywords:} Generative AI in education, Virtual Classroom, Multi-agents System (MAS), Turn-takings.}
}

\newpage

\begin{center}
    \Large \textbf{TÓM TẮT} \normalsize\\[13pt]
\end{center}

% Nội dung tóm tắt bằng tiếng Việt
{\fontsize{12}{14}\selectfont
\textbf{Tóm tắt:} Môi trường giáo dục truyền thống, đặc biệt trong môn toán, thường không đáp ứng được nhu cầu về sự hỗ trợ cá nhân cho tất cả học sinh. Với sự phát triển mạnh mẽ của Generative AI, học sinh có cơ hội tiếp cận kiến thức một cách sâu rộng hơn thông qua việc tương tác với chatbot AI. Tuy nhiên, phần lớn các hệ thống hiện nay vẫn tập trung vào vai trò của AI như một người hướng dẫn, mà chưa khai thác đầy đủ các yếu tố tương tác xã hội vốn có trong các lớp học truyền thống. Điều này mở ra một hướng nghiên cứu tiềm năng: mô phỏng lớp học ảo với nhiều người tham gia — nơi học sinh không chỉ học kiến thức mà còn được phát triển kỹ năng giao tiếp, hợp tác. Hệ thống đa tác tử (Multi-Agent System – MAS) dựa trên các mô hình ngôn ngữ lớn (LLM) nổi lên như một giải pháp đầy hứa hẹn cho bài toán này, nhờ khả năng suy luận, ra quyết định và phối hợp linh hoạt giữa các tác tử. Trong bối cảnh lớp học ảo, các tác tử có thể được thiết kế với những vai trò đa dạng như bạn học hay giáo viên, cùng tương tác với học sinh thật để hướng tới mục tiêu học tập chung. Tuy nhiên, việc mô phỏng một lớp học đa thành phần không đơn giản, đặc biệt trong việc điều phối lượt nói và kiểm soát nội dung hội thoại sao cho tự nhiên và hiệu quả. Luận văn này không chỉ trình bày quy trình thiết kế các tác tử theo các nguyên tác sư phạm, được thực nghiệm trong môn Toán, mà còn đề xuất một cơ chế quản lý lượt nói hiệu quả, giúp nâng cao tính mạch lạc và tương tác trong lớp học ảo so với phương pháp trước đây. Kết quả thực nghiệm và đánh giá cho thấy các tác tử không chỉ thực hiện đúng vai trò của mình mà còn thể hiện khả năng chủ động trong việc xác định thời điểm phát biểu phù hợp, góp phần tạo nên một môi trường học tập tương tác, linh hoạt, mở ra trải nghiệm học tập hiệu quả cho học sinh.  
}

\textit{\textbf{Từ khóa:} Generative AI trong giáo dục, Lớp học ảo, Hệ thống đa tác tử (MAS), Quản lý lượt nói.}
}

